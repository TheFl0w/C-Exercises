%\RequirePackage[ngerman=ngerman-x-latest]{hyphsubst}
\documentclass[ngerman]{tudscrreprt}
\usepackage[german]{babel}
\usepackage{hyphenat}
\hyphenation{Mathe-matik wieder-gewinnen aus-geführt ver-einfachen}
\pagestyle{tudheadings}
\usepackage{helvet}
\usepackage{courier}
\renewcommand{\familydefault}{\sfdefault}
\fontfamily{phv}\selectfont
\usepackage{selinput}\SelectInputMappings{adieresis={ä},germandbls={ß}}
\usepackage{tudscrsupervisor}
\usepackage{listings}
\usepackage{enumitem}
\usepackage{amsmath}
\lstset{language=C, basicstyle=\ttfamily, keywordstyle=\color[HTML]{4271AE}\bfseries,
                stringstyle=\color[HTML]{F5871F}\ttfamily,
		showstringspaces=false,
                commentstyle=\color[HTML]{718C00}\ttfamily,
                morecomment=[l][\color{magenta}]{\#}}
\begin{document}
\faculty{Fakultät Informatik}
\department{Programmierkurs C/C++}
%
\section*{Boolsche Werte und Schleifen}
\begin{enumerate}
\item Gegeben seien folgende Ausdrücke.
\begin{lstlisting}[frame=single]
/* i) */
(2 <= 3)
/* ii) */
(3 * 4 < 12)
/* iii) */
int a = 10, b = 10;
(a >= b && b <= a)
/* iv) */
int a = 10, b = 9;
(a < b || b < a)
/* v) */
int a = 10, b = 9;
((a == b) || (b == a)) || (a == b++)
/* vi) */
int a = 10, b = 9;
((a == b) || (b == a)) || (a == ++b)
\end{lstlisting}
\begin{enumerate}
\item Geben Sie für die Ausdrücke i) bis vi) jeweils an, ob sie auf true oder false abgebildet werden.
\begin{labeling}{iiii}
\item [i] :
\item [ii] :
\item [iii] :
\item [iv] : 
\item [v] : 
\item [vi] :
\end{labeling}
\end{enumerate}
\pagebreak
\item Gegeben seien folgende Schleifen.
\begin{lstlisting}[frame=single]
/* i) */
for (int i = 0; i < 5; i++) {
/* .. */
}
/* ii) */
for (int i = 0; i < 5; i += 2) {
/* .. */
}
/* iii) */
int i = 2;
do {
/* .. */
} while (i < 2);
\end{lstlisting}
\begin{enumerate}
\item Geben Sie für die Ausdrücke i) bis iii) jeweils an, wie oft der Schleifenrumpf ausgeführt wird.
\begin{labeling}{iiii}
\item [i] :
\item [ii] :
\item [iii] :
\end{labeling}
\end{enumerate}
\end{enumerate}
\pagebreak
\section*{Funktionen}
\begin{enumerate}
\setcounter{enumi}{1}
\item Gegeben seien folgende Funktionen.
\begin{lstlisting}[frame=single]
unsigned long long f(unsigned n) {
    if (n == 0)
        return 1;
    else
        return n * f(n - 1);
}
unsigned g(unsigned a, unsigned b) {
    if (a == 0)
        return b;
    else
        return g(a - 1, b + 1);
}
unsigned h(unsigned n) {
    if (n == 0)
        return 0;
    else
        return n + h(n - 1);
}
\end{lstlisting}
\begin{enumerate}
\item Geben Sie die folgenden Funktionswerte an.
  \begin{enumerate}[label=\roman*)]
    \item f(5)
    \item g(4, 7)
    \item h(10)
  \end{enumerate}
\item Geben Sie für die Funktionen f, g und h jeweils ein mathematisches Äquivalent an.
\item Schreiben Sie die Funktion h so um, dass sie einen for-Loop statt Rekursion zur Berechnung verwendet um den Stack klein zu halten.
\item Schreiben Sie eine Funktion binom, die als Parameter $n$ und $k$ erhält. Sie soll als Ergebnis den Wert des Binomialkoeffizienten $\binom{n}{k}$ als Ganzzahl zurückgeben.
\item Überlegen Sie, wie man die Berechnung des Binomialkoeffizienten für große $k$ vereinfachen kann.
\end{enumerate}
\end{enumerate}
\end{document}